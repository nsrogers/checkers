\documentclass[12pt]{article}
\usepackage{amssymb}
\usepackage{amsmath}
\usepackage[margin=1in]{geometry}
\usepackage{listings}
\lstset{basicstyle=\footnotesize, numbers=left}
\begin{document}

\begin{titlepage}
\title{\huge \bf Checkers with AI}
\date{\today}
\author{Thomas White \\
    \and Nathan Rogers \\
    \and Daniel Shubin}
\maketitle
\begin{center}
    CMPS 112 - Comparative Programming Languages \\
    Final Project Report
\end{center}
\end{titlepage}

\section{Introduction:}
    \subsection{Overview:}
        Our project was to create a checkers engine in Python,
        that is playable by two AI's and humans. The two AI's
        were written in C++ and Haskell.
    \subsection{Why we chose C++, Haskell, and Python:}
        We chose Python as the front end implementation language
        because Python allows you to easily interface with other
        languages. Python is also a scripting language.
        It is also a language that we knew and could efficiently
        implement the checkers engine.
        For our first AI, we chose C++ since our team has a lot of
        experience with the language, and is a good contrast to functional
        programming since it is purely imperative.
        For our second AI, we chose Haskell because we had experience from
        the class. And it is a contrast to the imperative languages since
        it is purely functional.


\section{Calling C++ and Haskell From Python}
    \subsection{C++}
        \subsubsection{Introduction:}
            Python has a built in module called ctypes which allows the programmer
            to use C types. These are literally variables that are of the same types
            one would use in a C program. This module also allows one to load in compiled
            library files and call functions in that library. The return values from these
            functions are C types as well and things like Python Lists or Dictionaries must
            be converted into C types before being used in the python program. In our project
            we wrote an AI in C++ which is called from our main Python program. We pass in
            the board and also a pointer to a list of moves to be returned, more on this
            later.
        \subsubsection{Code Example}
            \lstinputlisting[language=Python, firstline=7, lastline=21]{../checkers.py}
        \subsubsection{How the code works}
            First we convert our board, which is a list of ints in python, to a C style array
            of ints. On lines 3 and 4, we initialize another C array of ints which will eventually
            be the returned move. On lines 5 and 6 we load a function from our precompiled library
            and tell ctypes explicitly what the return type is going to be. We then call the function
            passing in the board and the pointer to our move array. When we return from the function,
            we must convert the C style array of ints into a Python list (lines 8-14). And finally
            we return the AI's move.
        \subsubsection{Why we chose this}
            The foreign function interface ctypes is very well documented and easy to use.
            It gives the programmer the capabilities (speed) of C/C++ programming
            and still have the higher level Python features in the main code.
\end{document}

















